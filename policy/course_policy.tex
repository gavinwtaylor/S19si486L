%% LyX 2.2.1 created this file.  For more info, see http://www.lyx.org/.
%% Do not edit unless you really know what you are doing.
\documentclass[11pt,english]{article}
\usepackage{mathptmx}
\renewcommand{\familydefault}{\rmdefault}
\usepackage[T1]{fontenc}
\usepackage[latin9]{inputenc}
\usepackage{geometry}
\geometry{verbose,tmargin=1in,bmargin=1in,lmargin=1in,rmargin=1in}
\setlength{\parindent}{0bp}
\PassOptionsToPackage{normalem}{ulem}
\usepackage{ulem}

\makeatletter

%%%%%%%%%%%%%%%%%%%%%%%%%%%%%% LyX specific LaTeX commands.
%% Because html converters don't know tabularnewline
\providecommand{\tabularnewline}{\\}

\makeatother

\usepackage{babel}
\begin{document}
\begin{center}
  SI486L: Machine Learning and Data Science
\par\end{center}

\begin{center}
Course Policy, Spring AY19
\par\end{center}

\uline{Coordinator}: Assoc. Prof. Gavin Taylor, MI355, x3-6816, taylor@usna.edu\\

\uline{Course Description}: Machine Learning is the study of mathematically
making autonomous conclusions about new data given insight from
previously-seen data.  This course will cover a broad scope of machine
learning and data science problems, including supervised learning,
unsupervised learning, and reinforcement learning.  Techniques will include
parametric approaches, kernelized methods, and neural networks.  Students will
also be exposed to an introduction to learning theory and the mathematical
underpinnings of modern machine learning.\\

\uline{Credits}: 2-2-4\\

\uline{Student Outcomes}:\\

Graduates of the program will have an ability to:\\ 
\quad{}1.  Analysis. Analyze a complex computing problem and to apply principles of computing and other relevant disciplines to identify solutions.\\
\quad{}2.  Implementation. Design, implement, and evaluate a computing-based solution to meet a given set of computing requirements in the context of the program's discipline.\\
\quad{}3.  Communication. Communicate effectively in a variety of professional contexts. \\
\quad{}4.  Ethics. Recognize professional responsibilities and make informed judgments in computing practice based on legal and ethical principles.\\
\quad{}5.  Teamwork. Function effectively as a member or leader of a team engaged in activities appropriate to the program's discipline.\\

\quad{}CS-6.  Theory. Apply computer science theory and software development fundamentals to produce computing-based solutions.\\
\quad{}IT-6.  Requirements. Identify and analyze user needs and to take them into account in the selection, creation, integration, evaluation, and administration of computing based systems.\\

\uline{Textbook(s)}: Abu-Mostafa, Magdon-Ismail, and Lin.  \underline{Learning
From Data}\\

\uline{Extra Instruction}: Extra instruction (EI) is strongly encouraged
and should be scheduled by email with the instructor. EI is not a
substitute lecture; students should come prepared with specific questions
or problems.\\

\uline{Grading Policy and Collaboration}: The guidance in the Honor Concept of the
Brigade of Midshipmen and the Computer Science Department Honor Policy
must be followed at all times. See 
\texttt{www.usna.edu/CS/resources/honor.php}. Specific instructions
for this course:
\begin{itemize}
\item Homework: There will be occasional homework assignments.  These can be
  completed collaboratively.  In my other classes, part of the purpose of
  homework is to force students to confront topics they do not understand;
  there is much less of that in this class.  In contrast, this is an elective,
  for people nearing the end of their major; you know when you do and do not
  understand concepts, and you will be responsible for taking proper action
  even without prompting from homework.
\item Projects: There will be many projects.  These will be performed in
  groups of 2 or 3, and will end with an in-class demonstration.  You may use
  any source for help, and discuss them with anybody, but \textbf{all
  submitted work must be your group's}, and all help must be documented.
\item Exams: There will be two midterms and a comprehensive final.  Should a
  make-up exam be needed, inform the instructor at least one week in advance.
\item Participation: The class is designed as a project-heavy, collaborative
  experience.  Participation is graded so that I can properly reward those who
  are on board with this.  Letting your partner do the work will hurt you both
  on the exams, and in participation grades.
\end{itemize}

\uline{Classroom Conduct}: The section leader will record attendance
and bring the class to attention at the beginning and end of each
class. If the instructor is late more than 5 minutes, the section
leader will keep the class in place and report to the Computer Science
department office. If the instructor is absent, the section leader
will direct the class. Drinks are permitted, but they must be in reclosable
containers. Food, alcohol, smoking, smokeless tobacco products, and
electronic cigarettes are all prohibited. Cell phones must be silent
during class. \\
\\
\uline{Late Policy}: Projects will not be accepted late without a
\emph{really} good reason.  Homeworks may be submitted one day late, with a
30\% subtractive penalty.\\
\uline{Grading}: The breakdown of the final course grade will be:\\
\begin{itemize}
\item \textbf{25\% Final Exam} - the final exam will be cumulative.
\item \textbf{20\% Mid-term Exams} (2) - Mid-terms are written, with no
  practical component.  Bring a calculator, we do math in here.
\item \textbf{45\% Programming Projects} - Detailed instructions for the
electronic submission will accompany each project.  Much of our classtime will
be dedicated to these projects, but they are not intended to be completely
finished in class.
\item \textbf{10\% Homework and Participation} - The percentage of this which
  is homework will be determined based on the amount of homework I feel I have
  to give.  More homework, higher percentage.
\end{itemize}

\end{document}
